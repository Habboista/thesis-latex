\chapter*{Abstract}

This thesis is about single image depth estimation (SIDE), a subfield of computer vision, and interpretability of deep learning methods.

SIDE deals with the problem of reconstructing the geometry of a scene from a single image of it.
As of today, this is done by assigning to each image pixel a depth value corresponding to the depth of the object that forms that pixel in the camera coordinate system.

Deep learning models are known to be black-boxes, meaning that they are not understandable (\textit{interpretable}) by humans.
All "state of the art" approaches in SIDE are black-boxes.

This work has two objectives: to theoretically investigate interpretability of algorithms and to propose a methodology for achieving it in SIDE.
It is argued that there are upper limits to interpretability of algorithms solving certain tasks like SIDE, implying that a black-box component \textit{must} be employed in order to solve them.
For achieving more interpretable methods, deep learning has to be confined to sub-tasks.
SIDE is known to be decomposable into the following sub-tasks: (1) estimate depth maps of image patches and (2) merge them into a full prediction.
Performed experiments show how it is possible to reduce the responsibilities of the deep learning model in the patch-wise prediction by slightly changing and simplifying the learning problem.
These responsibilities are passed to the merging sub-task that has to be approached accordingly.
An interpretable depth fusion procedure is proposed to this purpose.

\thispagestyle{empty}
\mbox{}
\newpage
