%-------------------------------------------------------------------
\chapter{Introduction}
\label{c:intro}
%-------------------------------------------------------------------

%
% toOl: the % symbol can be used for comments like C-style //
%

\epigraph{``A design of any kind shows its real value when taking beyond its original limits.''}{\emph{Tom Jennings}}

Some few conventions (and an example of nested item list):

\begin{itemize}
\item text lines must have variable length;
\item newline must appear only after fullstops;
\item this is a tradeoff for using versioning tools:
\begin{itemize}
\item lines too long (full paragraphs) would nitroduce diffs that are too large;
\item fixed maximum line width (e.g., 80 columns) lead to programs to reorganize the paragraph to fit the width, causing again diffs that are too large.
\end{itemize}
\end{itemize}
%
You can use the percentage symbol to avoid a space between item list and this paragraph, so that the first line of this paragraph is not indented.

Example of line breaking follows\footnote{And this is an example of footnote.}.

Lorem ipsum dolor sit amet, consectetur adipiscing elit, sed do eiusmod tempor incididunt ut labore et dolore magna aliqua.
Ut enim ad minim veniam, quis nostrud exercitation ullamco laboris nisi ut aliquip ex ea commodo consequat.
Duis aute irure dolor in reprehenderit in voluptate velit esse cillum dolore eu fugiat nulla pariatur.
Excepteur sint occaecat cupidatat non proident, sunt in culpa qui officia deserunt mollit anim id est laborum.

Enumerated list:

\begin{enumerate}
\item Lorem ipsum dolor sit amet, consectetur adipiscing elit, sed do eiusmod tempor incididunt ut labore et dolore magna aliqua.
\item Ut enim ad minim veniam, quis nostrud exercitation ullamco laboris nisi ut aliquip ex ea commodo consequat.
\item Duis aute irure dolor in reprehenderit in voluptate velit esse cillum dolore eu fugiat nulla pariatur.
\item Excepteur sint occaecat cupidatat non proident, sunt in culpa qui officia deserunt mollit anim id est laborum.
\end{enumerate}

%-------------------------------------------------------------------
\section{First section}
\label{s:com-auto}
%-------------------------------------------------------------------

Le automobili moderne sono un complesso insieme di meccanica ed elettronica, ci sono infatti parecchie ECU (Electronic Control Unit) atte a controllare tutti i sottosistemi a bordo ed in particolare sistemi cruciali come iniezione e sistema frenante~\cite{nolte1}.

Le ECU sono collegate anche a sensori ed attuatori e sono interconnesse attraverso linee dati dedicate.
L'interazione di tutte le ECU permette di avere una visione globale aggiornata del sistema sotto controllo, in questo caso l'automobile e di migliorare la qualit� del controllo stesso, l'affidabilit�, la sicurezza e il comfort.
Inoltre, essendo dispositivi elettronici, condividono la stessa linea di alimentazione e questo rende interessante una ricerca in questo ambito, ancora del tutto estraneo ad un sistema di comunicazione su powerline.
Nell'automotive, se si escludono i sottosistemi di tipo infotaintment (sistemi di intrattenimento multimediale), si impiegano attualmente tre tipi di bus\footnote{Una trattazione pi� approfondita verr� affrontata nel \cref{cap:fieldbus}} comunemente utilizzati: LIN~\cite{lin1}, CAN, FLEXRAY~\cite{flexray1}.

Ci sono molte ragioni che motivano la ricerca di una soluzione PLC in ambito automotive, essenzialmente si cerca di ridurre il cablaggio presente.
Il complesso insieme di cavi � il secondo fattore di costo e peso in un'autovettura; si pensi infatti che il sistema di cablaggio all'interno di una moderna automobile pu� raggiungere i 2 Km di lunghezza~\cite{bbfan1}.
Un elevato cablaggio implica:

I requisiti di comunicazione nei veicoli elettrici ed ibridi, sono simili a quelli dei motori a combustione interna e sono in relazione col crescente numero di connessioni tra le unit� principali~\cite{elec1}.

Le moderne ECU dedicano fino al 20\% della loro dimensione ai contatti e collegamenti fisici, perci� la riduzione del numero di fili comporta anche il ridimensionamento e la riduzione di costo di ogni singola ECU.

%----------------------------------------------------------
\section{Objectives of the thesis}
%----------------------------------------------------------

\lipsum[2-4]

%----------------------------------------------------------
\section{Organization of the document}
%----------------------------------------------------------

The thesis is organized as follows:

%
% toOl: notice that some labels are missing; thus in the generated pdf will show up the ?? symbols
% it is important to look for ?? symbols in the pdf to avoid this kind of issues
%
\begin{itemize}
\item \textbf{\cref{c:plc}} explains this
\item \textbf{\cref{c:fieldbus}} explains that
\item \textbf{\cref{c:hw}} presents this
\item \textbf{\cref{c:sw}} presents that
\item finally the conclusions in~\textbf{\cref{c:conc}}.
\end{itemize}

%----------------------------------------------------------
\section{Partnership}
%----------------------------------------------------------

Any possible additional information regarding the thesis.
